\href{https://travis-ci.com/pds2-dcc-ufmg/2019-1-grupo19}{\tt }

Um script simples para Ubuntu 18.\+04+ 
\begin{DoxyCode}
sudo add-apt-repository ppa:texus/tgui-0.8
sudo apt-get update
sudo apt-get install libsfml-dev libtgui-dev
\end{DoxyCode}


\#\# Como desenvolver localmente? 
\begin{DoxyCode}
# Gere os scripts de compilação para sua plataforma
cmake CMakeLists.txt
# Compile
make
# Execute
./graphODA
\end{DoxyCode}


O R\+E\+PL possui dois tipos de comando\+: atribuição e operação.

\#\#\#\# Atribuição de variável 
\begin{DoxyCode}
>>> var1 = var2
\end{DoxyCode}
 A variável {\itshape var1} passa a representar o grafo representado por {\itshape var2}.

\#\#\#\# import 
\begin{DoxyCode}
>>> var = import file\_name
\end{DoxyCode}
 A variável {\itshape var} passa a representar o grafo armazenado em {\itshape file\+\_\+name}. Se a leitura do arquivo falhar, {\itshape var} passa a representar um grafo vazio.

\#\#\#\# mst 
\begin{DoxyCode}
>>> var1 = mst var2
\end{DoxyCode}
 A variável {\itshape var1} passa a representar a {\bfseries árvore geradora mínima} do grafo representado por {\itshape var2}.

\#\#\#\# describe 
\begin{DoxyCode}
>>> var > describe
\end{DoxyCode}
 Exibe informações sobre o grafo representado por {\itshape var}.

\#\#\#\# show 
\begin{DoxyCode}
>>> var > show
\end{DoxyCode}
 Exibe o grafo representado por {\itshape var}.

\#\#\#\# edit 
\begin{DoxyCode}
>>> var > edit
\end{DoxyCode}
 Exibe o grafo representado por {\itshape var}, salvando as modificações efetuadas.

\#\#\#\# reaches 
\begin{DoxyCode}
>>> var > reaches a b
\end{DoxyCode}
 Verifica se o vértice {\itshape a} alcança o vértice {\itshape b} no grafo representado por {\itshape var}.

\#\#\#\# scc 
\begin{DoxyCode}
>>> var > scc
\end{DoxyCode}
 Exibe as componentes fortemente conexas do grafo representado por {\itshape var}.

\#\#\#\# shortest\+Path 
\begin{DoxyCode}
>>> var > shortestPath a b
\end{DoxyCode}
 Computa o peso do caminho mínimo do vértice {\itshape a} para o vértice {\itshape b} no grafo representado por {\itshape var}.

\#\#\#\# coloring 
\begin{DoxyCode}
>>> var > coloring
\end{DoxyCode}
 Computa uma coloração mínima para {\itshape var}. O algoritmo é polinomial quando {\itshape var} é cordal e {\bfseries T\+O\+DO} quando não é.

\#\#\#\# chromatic\+Number 
\begin{DoxyCode}
>>> var > chromaticNumber
\end{DoxyCode}
 Computa o número cromático, i.\+e. o menor número de cores necessárias para se colorir o grafo. O algoritmo é polinomial quando {\itshape var} é cordal e {\bfseries T\+O\+DO} quando não é.

\#\#\#\# greedy\+Coloring 
\begin{DoxyCode}
>>> var > greedyColoring
\end{DoxyCode}
 Computa uma coloração usando um algoritmo guloso na ordem dos vértices. Linear no tamanho do grafo.

\#\#\#\# maximum\+Clique\+Size 
\begin{DoxyCode}
>>> var > maximumCliqueSize
\end{DoxyCode}
 Computa o tamanho da clique máxima. Disponível apenas para grafos cordais.

\#\#\#\# art\+Points 
\begin{DoxyCode}
>>> var > artPoints
\end{DoxyCode}
 Encontra os vértices cujas remoções aumentam o número de componentes conexas do grafo.

\#\#\#\# bridges 
\begin{DoxyCode}
>>> var > bridges
\end{DoxyCode}
 Encontra as arestas cujas remoções aumentam o número de componentes conexas do grafo.

\#\#\#\# topo\+Sort 
\begin{DoxyCode}
>>> var > topoSort
\end{DoxyCode}
 Encontra uma ordenação topológica dos vértices do grafo.

\#\#\#\# center 
\begin{DoxyCode}
>>> var > center
\end{DoxyCode}
 Computa o centro da árvore {\itshape var}.

\#\#\#\# diameter 
\begin{DoxyCode}
>>> var > diameter
\end{DoxyCode}
 computa o diâmetro da árvore {\itshape var}. 